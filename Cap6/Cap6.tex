\chapter{Conclusions}\label{ch:conclusions}

\section{Summary}

Throughout this work, several techniques have been explored to develop a
model that can face \gls{ISS} of real world histopathology images in
crowdsourcing scenarios. Different approaches were studied and evaluated
to reach an optimal solution which could model the annotators' performance
from the input space. The following remarks are based on the results
obtained from the experiments conducted and reviewed literature:

\begin{itemize}
  \item Chained Gaussian Processes approaches are a robust approach
    for developing classification and regression tasks, however,
    their computational powerful requirements make them unsuitable
    for image segmentation tasks, specially since deep learning approaches
    have shown to be more efficient in the task of extracting features
    from images.
  \item Existing loss functions in the state of the art were insufficient
    to model the annotators' performance across the input space, from
    which the need for a new loss function which fulfils the requirements
    of the task was identified.
  \item A novel loss function was proposed and evaluated, showing to
    be a good alternative to the existing loss functions in the state
    of the art, and providing an efficient way to optimize parameters
    for reaching capability to model the annotators' performance
    across the input space.
  \item The use of U-Shaped Deep Learning models as a building block
    for the proposed solution was a valid approach, as it allows for
    the model to learn the spatial relationships between the pixels
    in the image, and the annotations, and to use this information to
    make predictions about the annotations of new images.
  \item Chained Deep Learning approaches in combination with the
    proposed loss function were the best approach for the task, outperforming
    the state of the art in the task of \gls{ISS} of histopathology images.
\end{itemize}

\section{Future work}

As tools and frameworks for bayesian approaches like Chained Gaussian Processes
\footnote{Including GPFlow, and GPFlux.}
evolve, it is expected to see a decrease in computational
requirements and hence, making mixed models more appealing for the
task of \gls{ISS} of histopathology images.
In that sense, the use of \gls{CCGPMA} as a building block for the
proposed solution could be a good approach, as it is a powerful model
that can take advantage of the inter-dependencies between the annotators,
and the powerfulness of \glspl{CNN} to extract features from the images.