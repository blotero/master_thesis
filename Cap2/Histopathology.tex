\section{Digital histopathological images}

Digital histopathology represents a significant advancement in
medical imaging, where traditional glass slides containing tissue
samples are digitized using specialized scanning devices. This
transformation has revolutionized the field of pathology by enabling
remote diagnosis, computer-aided analysis, and digital archiving of
tissue samples <<CITE>>.

\subsection{Whole Slide Imaging (WSI)}

Whole Slide Imaging (WSI) is the process of digitizing entire glass
slides at high resolution, creating a digital representation that can
be viewed, analyzed, and shared electronically. Modern WSI scanners
use sophisticated optical systems that capture multiple fields of
view at high magnification, which are then stitched together to
create a seamless digital image <<CITE>>. These systems typically employ:

\begin{itemize}
  \item High-resolution objectives (20x to 40x magnification)
  \item Precise motorized stages for accurate slide positioning
  \item Automated focus systems to maintain image quality
  \item High-quality cameras with large sensor arrays
\end{itemize}

The resulting digital slides can reach sizes of several gigabytes,
containing billions of pixels that capture the microscopic details of
tissue samples <<CITE>>.

\subsection{Regions of Interest (ROI)}

In digital histopathology, \glspl{ROI} are specific areas within a
whole slide image that contain diagnostically relevant information.
These regions can be:

\begin{itemize}
  \item Manually annotated by pathologists
  \item Automatically detected using computer vision algorithms
  \item Defined based on specific tissue characteristics or abnormalities
\end{itemize}

\glspl{ROI} are particularly important for:
\begin{itemize}
  \item Focusing computational analysis on relevant areas
  \item Reducing computational complexity in automated systems
  \item Facilitating targeted diagnosis and research
  \item Enabling efficient storage and transmission of critical information
\end{itemize}

\subsection{Staining Techniques}

Histopathological analysis relies heavily on various staining
techniques to enhance the visibility of different tissue components
and cellular structures. The choice of staining method depends on the
specific diagnostic requirements and the type of tissue being examined.

\subsubsection{Hematoxylin and Eosin (H\&E)}

Hematoxylin and Eosin (H\&E) staining is the most widely used
technique in histopathology, particularly in breast cancer diagnosis
<<CITE>>. This staining method provides:

\begin{itemize}
  \item Hematoxylin: Stains cell nuclei blue/purple, highlighting
    nuclear morphology
  \item Eosin: Stains cytoplasm and extracellular matrix pink/red,
    revealing tissue architecture
\end{itemize}

The popularity of H\&E staining in breast cancer histopathology stems
from its ability to:
\begin{itemize}
  \item Clearly visualize tumor architecture and growth patterns
  \item Distinguish between different types of breast cancer
  \item Identify important diagnostic features like nuclear pleomorphism
  \item Assess tumor grade and stage
\end{itemize}

Beyond breast cancer, H\&E staining is extensively used in various
medical specialties including:
\begin{itemize}
  \item General pathology
  \item Dermatology
  \item Gastroenterology
  \item Neurology
  \item Oncology
\end{itemize}

\subsubsection{Special Stains}

In addition to H\&E, various special stains are used for specific
diagnostic purposes:

\begin{itemize}
  \item \textbf{Immunohistochemistry (IHC)}: Uses antibodies to
    detect specific proteins, crucial for:
    \begin{itemize}
      \item Subtyping breast cancers (ER, PR, HER2)
      \item Identifying tumor markers
      \item Determining treatment options
    \end{itemize}
  \item \textbf{Periodic Acid-Schiff (PAS)}: Highlights carbohydrates
    and basement membranes
  \item \textbf{Masson's Trichrome}: Distinguishes between collagen
    and muscle fibers
  \item \textbf{Silver stains}: Used for detecting microorganisms and
    nerve fibers
\end{itemize}

These specialized staining techniques complement H\&E by providing
additional diagnostic information that is crucial for accurate
diagnosis and treatment planning <<CITE>>.
