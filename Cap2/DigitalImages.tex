\section{Modern concept of digital image}

A digital image is a numerical representation of a visual scene,
captured through various imaging devices and stored in a computer.
From a mathematical perspective, a digital image can be represented
as a function $f(x,y)$ that maps spatial coordinates $(x,y)$ to
intensity values. In the discrete domain, this function is sampled at
regular intervals, creating a matrix of values known as pixels
(picture elements).

\subsection{Types of digital images}

\subsubsection{Grayscale images}
Grayscale images are the simplest form of digital images, where each
pixel represents a single intensity value. Mathematically, a
grayscale image can be represented as a 2D matrix $I$ of size $M
\times N$, where each element $I(i,j)$ represents the intensity at
position $(i,j)$. The intensity values typically range from 0 (black)
to 255 (white) in 8-bit images, or from 0 to 65535 in 16-bit images.

\subsubsection{Color images}
Color images extend the grayscale concept by representing each pixel
with multiple channels, typically Red, Green, and Blue (RGB). A color
image can be represented as a 3D matrix $I$ of size $M \times N
\times 3$, where $I(i,j,k)$ represents the intensity of the $k$-th
color channel at position $(i,j)$. Other color spaces like HSV (Hue,
Saturation, Value) or CMYK (Cyan, Magenta, Yellow, Key) are also
commonly used in different applications.

\subsubsection{Multispectral images}
Multispectral images capture information across multiple wavelength
bands beyond the visible spectrum. These images can be represented as
a 3D matrix $I$ of size $M \times N \times B$, where $B$ is the
number of spectral bands. Each band $I(i,j,b)$ represents the
intensity at position $(i,j)$ for the $b$-th spectral band. This
representation is particularly useful in medical imaging, remote
sensing, and scientific applications.

\subsubsection{3D images and volumetric data}
Three-dimensional images extend the concept of pixels to voxels
(volume elements). A 3D image can be represented as a 3D matrix $V$
of size $M \times N \times D$, where $D$ represents the depth
dimension. Each voxel $V(i,j,k)$ represents the intensity at position
$(i,j,k)$ in the 3D space. This representation is fundamental in
medical imaging (CT, MRI), scientific visualization, and computer graphics.

\subsection{Mathematical representations}

The mathematical foundation of digital images relies on several key concepts:

\begin{itemize}
  \item \textbf{Sampling}: The process of converting a continuous
    image into a discrete representation. According to the
    Nyquist-Shannon sampling theorem, the sampling frequency must be
    at least twice the highest frequency present in the image to avoid aliasing.

  \item \textbf{Quantization}: The process of converting continuous
    intensity values into discrete levels. The number of quantization
    levels determines the image's bit depth and affects its quality
    and storage requirements.

  \item \textbf{Resolution}: The number of pixels per unit length in
    an image, typically measured in pixels per inch (PPI) or dots per
    inch (DPI).

  \item \textbf{Dynamic range}: The ratio between the maximum and
    minimum measurable light intensities in an image, often expressed
    in decibels (dB).
\end{itemize}

The mathematical representation of a digital image can be expressed as:

\begin{equation}
  I(x,y) = \sum_{i=0}^{M-1} \sum_{j=0}^{N-1} f(i,j) \cdot \delta(x-i, y-j)
\end{equation}

where $I(x,y)$ is the digital image, $f(i,j)$ represents the
intensity values, and $\delta(x-i, y-j)$ is the Kronecker delta function.

For color images, the representation extends to:

\begin{equation}
  I(x,y) =
  \begin{bmatrix}
    I_R(x,y) \\
    I_G(x,y) \\
    I_B(x,y)
  \end{bmatrix}
\end{equation}

where $I_R$, $I_G$, and $I_B$ represent the red, green, and blue
channels respectively.
