
\section{Aims}\label{sec:objectives}

With the mentioned considerations in section \ref{sec:state_of_art}
in mind, this work proposes a novel approach for \gls{ISS} tasks in
medical images, which aims to train a model whose learning approach
is adaptive to the labeler performance. This is done by introducing a
loss function capable of inferring the best possible segmentation
without needing separate inputs about the labeler performance. This
loss function is designed to implicitly weigh the labelers based on
their performance, with the presence of an intermediate reliability
map allowing the model to learn from the most reliable labelers and
ignore the noisy labels. This approach differs from existing
\gls{CNN}-based segmentation models, as it does not require explicit
supervision of the labeler performance, making it more generalizable
and adaptable to different datasets and labelers.
\subsection{General Aim}

The main purpose of this work is to develop a novel approach for
\gls{ISS} tasks in medical images, which can adaptively infer the best
possible segmentation without needing separate inputs about the
labeler performance. This approach is expected to outperform the
segmentation performance of other state of the art approaches,
eliminate the need for explicit labeler supervision, and enhance
automation in medical image analysis.

%----------------------------------------------------------------------
\subsection{Specific Aims}

\begin{itemize}

  \item To develop a novel loss function for \gls{ISS} tasks in
    medical images, capable of inferring the best possible
    segmentation without needing separate inputs about the labeler
    performance.

  \item Introducing a tensor map which codifies the reliability of
    each labeler, allowing the model to implicitly weigh the labelers
    based on their performance across the mask and classes space.

  \item To develop and test a deep learning model for \gls{ISS} tasks
    in medical images, which can learn from inconsistent labels and
    improve the segmentation performance compared to other solutions
    in state of the art.

\end{itemize}
