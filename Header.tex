\newcommand{\thesistitle}{Medical image segmentation in a multiple labelers
context: Application to the study of histopathology}
\newcommand{\thesistitlees}{Segmentación de imágenes médicas en un
contexto de múltiples anotadores: Aplicación al estudio de histopatologías}

% Packages
\usepackage[pagebackref=true]{hyperref}
%\usepackage{hyperref}
\usepackage[subfigure]{tocloft}
%\usepackage[subfig]{tocloft}
%\usepackage{tocloft}
%\usepackage{tocloft}
\usepackage[utf8]{inputenc}
\usepackage[english]{babel}
\usepackage{fancyhdr}
\usepackage{epsfig}
\usepackage{epic}
\usepackage{eepic}
\usepackage{threeparttable}
\usepackage{amscd}
\usepackage{here}
\usepackage{graphicx}
\usepackage{lscape}
%\usepackage{subfigure}
\usepackage{longtable}
\usepackage{setspace}
\usepackage{array}
\usepackage{float}
\usepackage{booktabs}
\usepackage{svg}
\usepackage{minted}
\usepackage{listings}
\usepackage{xparse}
\usepackage{xstring}
\usepackage[titletoc]{appendix}
%\usepackage{subfig}
\usepackage{enumitem}
\usepackage{cancel}
\usepackage{multirow}
\usepackage{longtable}
%\usepackage{hyperref}
\usepackage{tcolorbox}
\usepackage{subfig}
\usepackage{breakcites}

\raggedbottom

% Show margins
% \usepackage{showframe}
% \renewcommand*\ShowFrameLinethickness{0.1pt}
% \renewcommand*\ShowFrameColor{\color{black!10!white}}

\usepackage{caption}
%\usepackage{subcaption}

% Math
\usepackage{amstext,amsbsy,amsopn,amsmath,eucal,amsfonts,amssymb,amsthm}

% Vector commands
\newcommand{\vecx}{\mathbf{x}}
\newcommand{\vecy}{\mathbf{y}}

% Fonts
\usepackage{fontspec}
\setmainfont[
  Path=Fonts/Ancizar/,
  BoldFont={Ancizar-Serif-Bold.otf},
  ItalicFont={Ancizar-Serif-Regular-Italic.otf},
  BoldItalicFont={Ancizar-Serif-Bold-Italic.otf},
  %  SmallCapsFont={Ancizar-Serif-Light.otf},
  %  SlantedFont={Ancizar-Serif-Light.otf},
  %  UprightFont={Ancizar-Serif-Light.otf}
]{Ancizar-Serif-Regular.otf}
\setmonofont[
  Path=Fonts/Noto Sans Mono/,
  %  Scale=0.5
]{NotoSansMono-Regular.ttf}

\DeclareOldFontCommand{\rm}{\normalfont\rmfamily}{\mathrm}
\DeclareOldFontCommand{\sf}{\normalfont\sffamily}{\mathsf}
\DeclareOldFontCommand{\tt}{\normalfont\ttfamily}{\mathtt}
\DeclareOldFontCommand{\bf}{\normalfont\bfseries}{\mathbf}
\DeclareOldFontCommand{\it}{\normalfont\itshape}{\mathit}
\DeclareOldFontCommand{\sl}{\normalfont\slshape}{\@nomath\sl}
\DeclareOldFontCommand{\sc}{\normalfont\scshape}{\@nomath\sc}

% Titles
% \fontsize{<size>}{<bskip>}
% \titleformat{command}[shape]{format}{label}{sep}{before}[after]
\usepackage{titlesec}
% \titleformat{\chapter}{\normalfont\fontsize{200}{30}\bfseries}{\thechapter}{1em}{}
\titleformat{\section}{\normalfont\fontsize{20}{20}\bfseries}{\thesection}{1em}{}
\titleformat{\subsection}{\normalfont\fontsize{17}{17}\bfseries}{\thesubsection}{1em}{}
\titleformat{\subsubsection}{\normalfont\fontsize{14}{14}\bfseries}{\thesubsubsection}{1em}{}

% \titlespacing{command}{left spacing}{before spacing}{after spacing}[right]
\titlespacing{\section}{0pt}{2em}{1em}
\titlespacing{\subsection}{0pt}{2em}{1em}
\titlespacing{\subsubsection}{0pt}{1em}{0.5em}

% UN colors
% http://identidad.unal.edu.co/identidad-visual/b-directrices-y-especificaciones/b1-elementos-de-identidad-visual/#c1998
\usepackage{xcolor}
\definecolor{UN_primary}{RGB}{148,180,59}
\definecolor{UN_secondary}{RGB}{166,28,49}
\definecolor{UN_comp1}{RGB}{70,107,63}
\definecolor{UN_comp2}{RGB}{118,35,47}
\definecolor{UN_comp3}{RGB}{86,90,92}
\definecolor{UN_comp4}{RGB}{177,178,176}
% \definecolor{Nessa}{RGB}{1,87,155}

% Inline source code
\NewDocumentCommand{\quottable}{m}{\texttt{\bfseries\textcolor{UN_secondary}{#1}}}
\NewDocumentCommand{\quot}{m}{\texttt{\fontsize{\footnotesize}{\footnotesize}\bfseries\textcolor{UN_secondary}{#1}}}

% Block of Python code
\BeforeBeginEnvironment{minted}{\medskip}
\AfterEndEnvironment{minted}{\medskip}
\newminted[python]{python}{
  mathescape=true,
  xleftmargin=1cm,
  fontsize=\scriptsize,
  baselinestretch=1.2,
  python3=true,
  %  samepage=true,
  %  linenos=true,
  %  highlightlines={1,2-3,5-10},
  style=emacs
}

% Remove \fcolorbox inside the minted environment
\AtBeginEnvironment{minted}{\dontdofcolorbox}
\def\dontdofcolorbox{\renewcommand\fcolorbox[4][]{##4}}

% Cite in footer
\NewDocumentCommand{\footcite}{mm}{\textit{#1}\footnote{\href{#2}{#2}}}

% Tables font size
\AtBeginEnvironment{tabular}{
  \scriptsize
}

\AtBeginDocument{

  % Rename lists
  % \renewcommand{\contentsname}{\hfill\normalfont\Large CONTENTS}
  \renewcommand{\listfigurename}{\hfill\normalfont\Large LIST OF FIGURES}
  \renewcommand{\listtablename}{\hfill\normalfont\Large LIST OF TABLES}

  % Labels
  \renewcommand{\theequation}{\thechapter-\arabic{equation}}
  \renewcommand{\thefigure}{\textbf{\thechapter-\arabic{figure}}}
  \renewcommand{\thetable}{\textbf{\thechapter-\arabic{table}}}
  \renewcommand{\thesubfigure}{\alph{subfigure}}

  % Justify without hyphenation
  \tolerance=1
  \emergencystretch=\maxdimen
  \hyphenpenalty=10000
  \hbadness=10000
}

% Table of contents depth
\setcounter{tocdepth}{2}

% Title numeration depth
\setcounter{secnumdepth}{2}

% Captions
\usepackage{caption}
\DeclareCaptionLabelSeparator{none}{ }
\captionsetup{labelsep=none, font=small}

\newcommand\quotcaption[1]{
  \captionsetup{font=footnotesize}
\caption{#1}}

% Bibliography with backref
\renewcommand*{\backrefalt}[4]{\hspace*{\fill}\normalsize{
    \ifcase #1 (Not cited.)   % not cited
    \or (page~#2)             % cited once
    \else (\mbox{pages~#2})   % cited multiple pages
    \fi
}}

% Link colors
\hypersetup{
  colorlinks=true,
  linkcolor=UN_comp1,
  urlcolor=UN_comp1,
  filecolor=magenta,
  citecolor=UN_secondary,
  pdftitle={\thesistitle - Brandon Lotero Londoño}
}

% Abbreviations
\usepackage[acronym=true,
  toc=true,
  numberline=false,
  nopostdot=true,
  section=chapter,
nomain=true]{glossaries-extra}
\setabbreviationstyle[acronym]{long-short}
\renewcommand{\glsnamefont}[1]{\textsc{\normalfont\large\bfseries #1}}
\glssetcategoryattribute{acronym}{hyperoutside}{false}

% Fonts used in table of contents
\renewcommand{\cfttoctitlefont}{\normalfont\ssfamily\Large}
% \renewcommand{\cftpartfont}{\bfseries}
\renewcommand{\cftchapfont}{\normalfont\ssfamily\bfseries}
\renewcommand{\cftsecfont}{\normalfont\ssfamily}
\renewcommand{\cftsubsecfont}{\normalfont\ssfamily}
\renewcommand{\cftsubsubsecfont}{\normalfont\ssfamily\small}

% Fancy chapters
\usepackage[Bjarne]{fncychap}
\ChRuleWidth{1pt}
\ChNameUpperCase\ChNameVar{\raggedleft\normalsize\rm}
\ChTitleUpperCase\ChNumVar{\raggedleft \bfseries\Large}
\ChTitleVar{\raggedleft \Large\rm}

% Page margins
\usepackage{geometry}
\geometry{
  inner=37.125mm,
  outer=33.4125mm,
  top=37.125mm,
  bottom=37.125mm,
  headsep=24pt,
  footnotesep=24pt,
  marginparwidth=50pt,
}

% Page Style
\pagestyle{fancyplain}
% \renewcommand{\chaptermark}[1]{\markboth{\thechapter\; #1}{}}
\renewcommand{\chaptermark}[1]{\markboth{#1}{}}
\renewcommand{\sectionmark}[1]{\markright{\thesection\; #1}}
\lhead[\fancyplain{}{\thepage}]{\fancyplain{}{\rightmark}}
\rhead[\fancyplain{}{\leftmark}]{\fancyplain{}{\thepage}}
\fancyfoot{}
\thispagestyle{fancy}

% space between paragraph and preceding text
\setlength{\parskip}{1em plus1em minus1em}

% paragraph indentation
\setlength{\parindent}{0em}

% line spacing
\renewcommand{\baselinestretch}{1.3}

% Linespace description
\AtBeginEnvironment{description}{\linespread{0}\small\selectfont}

% Change fontsize
\renewcommand{\UrlFont}{\ttfamily\footnotesize}

\newsavebox\mtbox
\newcommand{\mtgraphics}[2][]{%
  \sbox\mtbox{\includegraphics[#1]{#2}}%
  \null%
  \dimen255=\dimexpr\pagegoal-\pagetotal-\baselineskip\relax%
  \ifdim\dimen255>\ht\mtbox
  \usebox\mtbox
  \else
  \dimen254=\dimen255%
  \dimen255=\dimexpr\ht\mtbox-\dimen254\relax%
  \includegraphics[trim=0mm 1.\dimen255 0mm 0mm, clip=true, #1]{#2}
  \par
  \includegraphics[trim=0mm 0mm 0mm \dimen254, clip=true, #1]{#2}
\fi}

\usepackage{pdfpages}

\newcommand{\ssfamily}{\sffamily}

\tcbuselibrary{many}
\tcbset{
  enhanced,
  breakable,
  attach boxed title to top left={
    xshift=0.5cm,
    yshift= -3.5mm, % What do I put here? I'd like to have something like:
    %       yshift= -0.5\titleboxheight
  },
  top=4mm,
  coltitle=black,
  beforeafter skip=\baselineskip,
}

\newenvironment{mybox}[1]{%
  \begin{tcolorbox}[title={#1}]%
  }{
  \end{tcolorbox}
}

\usepackage{bm}
\providecommand{\mat}[1]{{\bm{#1}}}
\providecommand{\ve}[1]{{\bm{#1}}} %vector format
\providecommand{\mat}[1]{{\bm{#1}}} %matrix format
\providecommand{\promed}[1]{\ensuremath{\mathds{E}\left\lbrace#1\right\rbrace}}
%Expected value
\providecommand{\norm}[1]{\ensuremath{\left\lVert#1\right\rVert}} %l-norm
\providecommand{\oper}[1]{{\operatorname{#1}}\!} %operator name
\providecommand{\Real}{\displaystyle\mathbb{R}} %R
\providecommand{\N}{\displaystyle\mathbb{N}} %N
\providecommand{\abs}[1]{\ensuremath{\left\lvert#1\right\rvert}} %||
\DeclareMathOperator{\en}{\displaystyle\negthinspace\in\negthinspace} %∈
\DeclareMathOperator{\igual}{\displaystyle\negthinspace=\negthinspace} %=
\DeclareMathOperator{\x}{\displaystyle\negthinspace\times\negthinspace} %x
\DeclareMathOperator{\tire}{\displaystyle\negthinspace--\negthinspace} %
\providecommand{\s}[1]{\negthickspace#1\negthickspace} %
\DeclareMathOperator*{\argmax}{argmax} %argmax

\captionsetup{format=plain}